The vehicle industry is one of the biggest in the world with billions of revenues every year and this makes it preferable for fraudulent
parties and individuals to conduct their suspicious activities. These activities can be observed through several types of known methods such
as: copycats who create counterfeited replacement parts and sell them as originals, dealers who modify vehicles to hide known malfunctions;
for example: adding an electronic resistance to manipulate vehicle unit to register missing parts like the air bag which can endanger
people's life. Also, modifying the vehicle’s odometer to show a lower number and hiding relative information is from buyers to increase
vehicle value at the cost of the safety of others. Aside from endangering peoples, these acts lead to huge financial losses which emphasizes
the need for creating solutions to eliminate these challenges.

For an individual to protect himself from previous frauds when purchasing a vehicle, in the traditional way, he needs to acquire information
about the vehicle history. By nature, vehicles history is segregated into several categories, namely: accidents records, maintenance logs,
exports and imports, and ownership records and some of this information is stored by country and others by shop or dealer which makes the
mission difficult and sometimes impossible to accomplish. To overcome this issue, several solutions were raised, such as Carseer and Carfax
which are companies that collect vehicle information's across different countries and sell them as a report to individuals. Even though
these solutions are somewhat effective, they have two major drawbacks, for instance: the adaptation from related parties and the fact they
are a centralized organization, and their databases could be altered or hacked. To conclude, the is a critical need for a better solution to
ensure trust and integrity between buyers and sellers.

Blockchain is a distributed, decentralized, and immutable digital ledger that is used to store data and transactions in a chain of connected
blocks. Each block is hashed to ensure immutability and is connected to the chain using the hash of its previous block. Additionally, to
ensure decentralization and distribution, all nodes in a BC network have a copy of the ledger and depending on the network type, either all
nodes contribute to transactions and block validation or only a set of selected nodes. This characteristic makes it an efficient candidate
for systems that facilitate interactions and transactions between trustless parties without the need for centralized entities to manage the
system. For example: E-voting
\cite{AlMaaitah2021}\cite{al2022blockchain}, drugs \cite{al2022blockchain}\cite{AlShorman2018}, health insurance \cite{Alhasan2021}, internet of things
\cite{Qatawneh2020}, and used
vehicle market [8].
Since the release of Bitcoin in 2009, blockchain has gained huge popularity among researchers in the academic spectrum. Consequently,
there is a plethora of research volume about it is implementation in the auto industry. The research in this subject can be categorized in 4
categories namely: vehicle registration and secure purchase, tracking used vehicle history, securing the supply chain, and preventing
counterfeiting. In \cite{Tran2021-fh} proposed a vehicle registration system using BC to improve the process of registering
vehicles and facilitate the interactions and communications between the vehicle owner and the registration office. Even though the whole is
done using the system, after the registration office approves the request, the owner is required to bring the vehicle for a physical
inspection to complete the process and receive the license plate. In another research, \cite{Khandelwal2021-to} proposed a BC system
to secure the order of new vehicles using Ethereum network. The system uses two smart contracts to facilitate the operation, one between the
customer and the dealer called CustomerDealer and one between the dealer and manufacturer called DealerManufacturer, the smart contracts
handle the information communication between parties as well as the payment of the vehicle price.


\cite{Yu2021-yq}
proposed a BC framework to prevent fraud in the used vehicle market, also it is designed to be led by the government with other
departments to add required information to the ledger. Although this design decision decreases the framework decentralization, it can
enforce regulations and rules on stakeholders to ensure network integrity. Furthermore, it includes the intermediaries who facilitate the
buying and selling between the owner and the buyer for a predetermined commission. \cite{Zhang2021-jo} proposed a BC system
based on HL consists of six layers to trace and log vehicle history. The layers are user layer which contains the UI, application layer
which contains the system portal, contract layer containing the smart contracts, network layer includes the network protocols, data layer,
and the acquisition layer. Finally, it provides users with the ability to obtain vehicle reports using a QR code attached to the vehicle
which can enhance the system privacy by limiting the report access to only potential buyers.


\cite{Yahiaoui2020-gp}
proposed a supply chain model for the automotive sector using BC to coordinate the tracking of automotive products. As in a
traditional supply chain model, logistics information is only available to the entities involved in the actual process without providing the
end-user with a method to validate the origin of the product. But with the implementation of a decentralized ledger, the authors can provide
the end-users with the ability to validate all information's relevant to the product. The model uses smart contracts to facilitate the
transactions between stakeholders with the aim of overcoming data barriers and improving the traditional supply chain system in several
aspects such as security, information sharing, and system integration. \cite{Lu2019-wk} Proposed a supply chain model based on
HL fabric BC to prevent the vehicle parts counterfeiting. The model introduced the concept of a self-aware vehicle with traceable parts,
with the vehicle itself as an independent stakeholder that can validate the parts mounted on it using a unique identifier for each part and
add transactions to the chain to report the state of parts. Even though malicious parties might try to duplicate the id from original parts
or use the id of parts from salvaged or recalled vehicle, the supplier is the only entity that has the privilege to add new parts to the
network such acts are prevented. Authors also stated some challenges that can face the model such as the data privacy and which information
should be considered as private to protect stakeholders from other competitors, and the communication security in the network which can be
solved using TLS to secure all communications.


\cite{Reimers2019-el}
proposed a supply chain prototype that integrate BC with IoT to provide end users with a report for vehicle shipments. Their goal is
to provide a PoC for the integration of IoT devices such as raspberry Pi devices with NFC tags and sensors that is attached to the vehicle,
the raspberry Pi’s are attached to MQTT using Node-Red to facilitate the communication between IoT devices and the ledger. Sensors and IoT
devices data are uploaded to the chain and when the vehicle arrives, the users can access the vehicle report to validate the vehicle status.
\cite{Ada2021-fn} proposed a BC supply chain architecture for an existing organization after identifying several challenges it faces
with traditional supply chain. Authors stated that it suffered from maintaining the Inventory Quality Ratio (IQR), so they apply a
multi-layered architecture with HL to connect various stakeholders. The prototype was simulated with Anylogic and the simulation results
show that it can bring a significant value to the organization as the numbers shows that the maximum limit of mean daily costs and mean
waiting time for factory, wholesaler, and retailer after employing the BC has significantly decreased with minimum retailer stock level and
maximum retailer stock level that helps to measure IQR. \cite{Zafar2022-uk} proposed a BC system to manage supply chain for automotive
industry using HL fabric. The system addresses three stages of the supply chain: manufacturing, sales, and maintenance and each vehicle is
identified using a unique identifier which is used on all transactions on the system. Finally, the authors stated that the system provides
better access control, transparency, encryption, and immutability over traditional supply chains and due to the permissioned BC, buyers can
only view the information on the ledger.

In \cite{Saldamli2020-zk}
proposed a model to secure vehicle data by integrating Ethereum BC and OpenXC devices mounted on the vehicle. OpenXC uses the OBD
II port on the vehicle to read and record data for the control unit, and the authors analyzed the data to detect several events such as:
driving behavior, vehicle maintenance, and odometer fraud check. \cite{Wang2020-tm} proposed a trust-driven framework for vehicle
product and service system (PSS) to improve trust and transparency among stakeholders by reducing the risk of scams and fraud. The framework
can integrate applications, and it contains three applications: Decentralized vehicle ledger to store vehicle data, automated valuation
which uses the data from the ledger to evaluate vehicle price, and insurance recommendations which uses the data from the ledger to provide
customized insurance policies based on the individual clients' requirements. \cite{Na2021-tc} proposed a lightweight BC
system that integrates sensors and IoT devices to vehicle to collect and protect vehicle image data from being forged. Each vehicle is a
node that encrypts images and creates transactions to upload them to an InterPlanet File System (IPFS) which is a distributed storage
system.

In this research, we present a new BC framework based on Hyperledger (HL) fabric to ensure trust and prevent common counterfeiting and fraud
methods in the vehicle market. Hyper-ledger Fabric is a permissioned BC framework developed for enterprise use. It provides a modular and
scalable architecture that allows organizations to build private, secure, and customizable BC networks, enabling trusted transactions and
chaincodes execution within a business network. The framework will act as a log to store momentous events in the vehicle life cycle and
provide it to potential buyers in a report-form. This ensures that the buyer is fully informed and aware of the vehicle's real status to
prevent malicious parties from taking advantage of buyers for their own benefit.

The rest of the paper will be structured as follows: the Method section will explain the framework architecture and methods,
Results and discussion section will describe our evaluation results, and
finally, the Conclusion section will discuss the results and the potentials for future research.

